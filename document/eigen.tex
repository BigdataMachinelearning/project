\chapter{Sparse Matrix}
\begin{enumerate}
\item class Triplet
\begin{itemize}
\item Triplet() : m$\_$row(0), m$\_$col(0), m$\_$value(0)
\item Triplet(const Index$\&$ i, const Index$\&$ j, const Scalar$\&$ v = Scalar(0))
\item const Index $\&$ col () const
\item const Index $\&$ 	row () const
\item const Scalar $\&$ value () const
\end{itemize}
\item Sparse Matrix
\\用稀疏的方式表达矩阵,
\begin{itemize}
\item binaryExpr (const SparseMatrixBase$<$ OtherDerived $> \&$other,
const CustomBinaryOp $\&$func = CustomBinaryOp() )	
\\二值操作, Eigen有一系列的二值运算,输入一个Eigen对象,逐个元素进行
运算,返回一个相应的运算对象
\item size()
\\返回矩阵大小,这里的大小是矩阵的真正大小,不是压缩后的大小。
\item resize(int row, int col)
\\这里的resize和stl的resize不太一样,resize的结果是矩阵的真实大小,
不是存储的大小,因此size()返回的就是resize()时候设置的
\item 元素访问,
SpMat有两层,外层和内层,如果是列优先,外层指的是每列的起始
位置,内层指的是低维。
\\innerSize()返回内部维数大小, cols()返回列数。
innerSize()在列优先矩阵里面相当于rows()
\item setFromTriplets(beg, end);
\\SpMat很重要的一个函数,输入是三元组,运行该函数前需要首先进行
resize(), 返回一个压缩矩阵
\item reserve()
\\预留空间,在insert前使用,预告估计要分配的元素,
可以使得insert的时间最快。
\item insert(rows, cols)
\\返回一个引用,用于存放值,在不重新分配空间的情况下,
可以在对数时间完成,如果是排好序的,可以在常数时间完成?
eigen的默认情况是,插入一个新值,矩阵不再是压缩形式,预留两个
空间。
\end{itemize}
\end{enumerate}
